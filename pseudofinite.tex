\documentclass[10pt,oneside, openany]{book}
\usepackage[utf8]{inputenc}
\usepackage[T1]{fontenc} 
% \usepackage[dvips,paperheight=11cm, paperwidth=21cm, ,hmargin={9.5cm,0.5cm},vmargin={0.5cm,0.5cm}]{geometry}
\usepackage[a4paper, hmargin={4cm,4cm},vmargin={2.5cm,2.5cm}]{geometry}
\usepackage[unicode, colorlinks=true,bookmarksopen=false,linkcolor=blue,citecolor=red]{hyperref}
\usepackage{subfiles}
\usepackage{calc}
\usepackage{comment}
\usepackage{amssymb}
\usepackage{amsthm}
\usepackage{amsmath}
\usepackage{amsrefs}
\usepackage{titlesec}
\usepackage{titletoc}
\usepackage{dsfont}
\usepackage{euscript}
\usepackage{fourier-orns}
%\usepackage{pxfonts}
%\usepackage{newpxtext}
%\usepackage{tgpagella}
\usepackage{palatino}
\usepackage[sc]{mathpazo} % add possibly `sc` and `osf` options
%\usepackage{eulervm}
%\usepackage[adobe-utopia]{mathdesign}
\usepackage{graphicx}
\usepackage{pgfplots}
\usepgfplotslibrary{polar}
\usetikzlibrary{shapes.geometric}
\usetikzlibrary{calc}
\pgfplotsset{width=10cm,compat=1.9}
\usepackage{tikz}
\usepackage{tikz-cd}
\tikzcdset{
arrow style=tikz,
diagrams={>=latex}
}


\linespread{1.2}
\setlength{\parindent}{0ex}
\setlength{\parskip}{.4\baselineskip}
\definecolor{blue}{rgb}{0, 0.1, 0.6}

\DeclareFontFamily{OT1}{pzc}{}
\DeclareFontShape{OT1}{pzc}{m}{it}{<-> s * [1.10] pzcmi7t}{}
\DeclareMathAlphabet{\mathpzc}{OT1}{pzc}{m}{it}


\newcommand{\mylabel}[1]{{\ssf{#1}}\hfill}
\renewenvironment{itemize}
  {\begin{list}{$\triangleright$}{%
   \setlength{\parskip}{0mm}
   \setlength{\topsep}{.4\baselineskip}
   \setlength{\rightmargin}{0mm}
   \setlength{\listparindent}{0mm}
   \setlength{\itemindent}{0mm}
   \setlength{\labelwidth}{2ex}
   \setlength{\itemsep}{.2\baselineskip}
   \setlength{\parsep}{1ex}
   \setlength{\partopsep}{0mm}
   \setlength{\labelsep}{1ex}
   \setlength{\leftmargin}{\labelwidth+\labelsep}
   \let\makelabel\mylabel}}{%
   \end{list}\vspace*{-1.3mm}}

\renewcommand*{\emph}[1]{%
   \smash{\tikz[baseline]\node[rectangle, fill=green!40, rounded corners, inner xsep=0.5ex, inner ysep=0.2ex, anchor=base, minimum height = 2.7ex]{#1};}}

\def\jj{\mbox{-}}
\def\E{\exists}
\def\A{\forall}
\def\mdot{\mathord\cdot}
\def\models{\vDash}
\def\pmodels{\mathrel{\models\kern-1.5ex\raisebox{.5ex}{*}}}
\def\notmodels{\nvDash}
\def\proves{\vdash}
\def\notproves{\nvdash}
\def\proves{\vdash}
\def\provesT{\mathrel{\mathord{\vdash}\hskip-1.13ex\raisebox{-.5ex}{\tiny$T$}}}
\def\ZZ{\mathds Z}
\def\NN{\mathds N}
\def\QQ{\mathds Q}
\def\RR{\mathds R}
\def\BB{\mathds B}
\def\CC{\mathds C}
\def\PP{\mathds P}
\def\Indicator{\mathds I}
\def\Ar{{\rm Ar}}
\def\dom{\mathop{\rm dom}}
\def\range{\mathop{\rm range}}
\def\supp{\mathop{\rm supp}}
\def\rank{\mathop{\rm rank}}
\def\dcl{\mathop{\rm dcl}}
\def\acl{\mathop{\rm acl}}
\def\rad{\mathop{\rm rad}}
\def\eq{{{\rm eq}}}
\def\ccl{{\rm ccl}}
\def\Th{\textrm{Th}}
\def\Diag{{\rm Diag}}
\def\atpmTh{\textrm{Th}_\atpm}
\def\Mod{\mathop{\rm Mod}}
\def\Indicator{{\mathds I}}
\def\Rmod{{\mbox{\scriptsize $R$-mod}}}
\def\Aut{{\rm Aut\kern.0ex}}
\def\Def{{\rm Def\kern.0ex}}
\def\Brl{{\rm Brl\kern.0ex}}
\def\Autf{\mathord{\rm Aut\kern.15ex{f}\kern.15ex}}
\def\orbit{\O}
\def\oorbit{\mathpzc{o}}
\def\oorbitf{\mathpzc{of\!}}
\def\id{\textrm{id}}
\def\tp{{\rm tp}}
\def\atpm{{\tiny\rm at^\pm}}
\def\qftp{\textrm{qf\mbox{-}tp}}
\def\attp{\textrm{at\mbox{-}tp}}
\def\atpmtp{\atpm\mbox{-}\textrm{tp}}
\def\Deltatp{\Delta\mbox{-}\textrm{tp}}
\def\pmDelta{\Delta\hskip-.3ex\raisebox{1ex}{\tiny$\pm$}}
\def\pmDeltatp{\noindent\pmDelta\hskip-.3ex\textrm{-tp}}
\def\EMtp{\textrm{{\small EM}\mbox{-}tp}}


\def\sm{\smallsetminus}
\def\atpmL{L_{\atpm}\hskip-.3ex}
\def\qfL{L_{\rm qf}}
\def\atL{L_{\tiny\rm at}\hskip-.3ex}
\def\simdiff{\mathop\vartriangle}
\def\IMP{\Rightarrow}
\def\PMI{\Leftarrow}
\def\IFF{\Leftrightarrow}
\def\NIFF{\nLeftrightarrow}
\def\imp{\rightarrow}
\def\pmi{\leftarrow}
\def\iff{\leftrightarrow}
\def\niff{\mathrel{{\leftrightarrow}\llap{\raisebox{-.1ex}{{\small$/$}}\hskip.5ex}}}
\def\nequiv{\mathrel{\mbox{$\equiv$\llap{{\small$/$}\hskip.3ex}}}}
\def\equivEM{\stackrel{\smash{\scalebox{.5}{\rm EM}}}{\equiv}}
\def\equivL{\stackrel{\smash{\scalebox{.5}{\rm L}}}{\equiv}}
\def\equivKP{\stackrel{\smash{\scalebox{.5}{\rm KP}}}{\equiv}}
\def\equivSh{\stackrel{\smash{\scalebox{.5}{\rm Sh}}}{\equiv}}
\def\dIFF{\IFF\hskip-2.2ex\smash{\raisebox{1.3ex}{\tiny def}}}
\def\deq{\mathrel{=\hskip-1.9ex\smash{\raisebox{1.2ex}{\tiny def}}}}
\def\swedge{\mathbin{\raisebox{.2ex}{\tiny$\mathbin\wedge$}}}
\def\svee{\mathbin{\raisebox{.2ex}{\tiny$\mathbin\vee$}}}

\def\bigsum{\mathop{\mbox{\large$\displaystyle\sum$}}}
\def\bigint{\mathop{\mbox{\large$\displaystyle\int$}}}


\def\hookdoubleheadrightarrow{\hookrightarrow\mathrel{\mspace{-15mu}}\rightarrow}

\def\P{\EuScript P}
\def\D{\EuScript D}
\def\Aa{\EuScript A}
\def\Ee{\EuScript E}
\def\X{\EuScript X}
\def\Y{\EuScript Y}
\def\Z{\EuScript Z}
\def\C{\EuScript C}
\def\U{\EuScript U}
%\def\H{\EuScript H}
\def\I{\EuScript I}
\def\V{\EuScript V}
\def\R{\EuScript R}
\def\F{\EuScript F}
\def\G{\EuScript G}
\def\B{\EuScript B}
\def\M{\EuScript M}
\def\Ll{\EuScript L}
\def\K{\EuScript K}
\def\O{\EuScript O}
\def\J{\EuScript J}
\def\S{\EuScript S}
\def\<{\langle}
\def\>{\rangle}
\def\0{\varnothing}
\def\theta{\vartheta}
\def\phi{\varphi}
\def\epsilon{\varepsilon}
\def\ssf#1{\textsf{\footnotesize #1}}

\titlecontents{section}
[3.8em] % ie, 1.5em (chapter) + 2.3em
{\vskip-1ex}
{\contentslabel{1.5em}}
{\hspace*{-2.3em}}
{\titlerule*[1pc]{}\contentspage}


\titleformat
{\chapter} % command
[display] % shape
{\bfseries\LARGE} % format
{Chapter \ \thechapter} % label
{0.5ex} % sep
{} % before-code
[] % after-code
\titleformat{\section}[block]{\Large\bfseries}{\makebox[5ex][r]{\textbf{\thesection}}}{1.5ex}{}
\titlespacing*{\chapter}{0em}{.5ex plus .2ex minus .2ex}{2.3ex plus .2ex}
\titlespacing*{\section}{-9.7ex}{3ex plus .5ex minus .5ex}{1ex plus .2ex minus .2ex}

\renewcommand*\thesection{\arabic{section}}

\newtheoremstyle{mio}% name
     {2\parskip}%      Space above
     {\parskip}%      Space below
     {\vl}%         Body font
     {}%         Indent amount (empty = no indent, \parindent = para indent)
     {\bfseries}% Thm head font
     {}%        Punctuation after thm head
     {1.5ex}%     Space after thm head: " " = normal interword space;
           %   \newline = linebreak
     {\llap{\thmnumber{\vl #2}\hskip2mm}\thmname{\vl #1}\thmnote{\kern1ex\bfseries\vl #3}}% Thm head spec (can be left empty, meaning `normal')

\newtheoremstyle{liscio}% name
     {2\parskip}%      Space above
     {0mm}%      Space below
     {}%         Body font
     {}%         Indent amount (empty = no indent, \parindent = para indent)
     {\bfseries}% Thm head font
     {}%        Punctuation after thm head
     {1.5ex}%     Space after thm head: " " = normal interword space;
           %   \newline = linebreak
     {\llap{\thmnumber{#2}\hskip2mm}\thmname{#1}\thmnote{\bfseries{} #3}}% Thm head spec (can be left empty, meaning `normal')



\newcounter{thm}[chapter]

% \renewcommand{\thethm}{\thechapter.\arabic{thm}}
\renewcommand{\thethm}{\arabic{thm}}
\theoremstyle{mio}
\newtheorem{theorem}[thm]{Theorem}
\newtheorem{corollary}[thm]{Corollary}
\newtheorem{proposition}[thm]{Proposition}
\newtheorem{lemma}[thm]{Lemma}
\newtheorem{fact}[thm]{Fact}
\newtheorem{definition}[thm]{Definition}
\newtheorem{assumption}[thm]{Assumption}
\newtheorem{void_thm}[thm]{\kern-1ex}
\theoremstyle{liscio}
\newtheorem{void_def}[thm]{\kern-1ex}
\newtheorem{remark}[thm]{Remark}
\newtheorem{notation}[thm]{Notation}
\newtheorem{note}[thm]{Note}
\newtheorem{exercise}[thm]{Exercise}
\newtheorem{example}[thm]{Example}
\setlength{\partopsep}{0mm}
\setlength{\topsep}{0mm}

\def\QED{\noindent\nolinebreak[4]\hfill\rlap{\ \ $\Box$}\medskip}
\renewenvironment{proof}[1][Proof]%
{\smallskip\begin{trivlist}\item[\hskip\labelsep {\bf #1}]}
{\QED\end{trivlist}}

\newenvironment{void}[1][]%
{\begin{trivlist}\item[\hskip\labelsep {\bf #1}]}
{\QED\end{trivlist}}



\pagestyle{plain}

\definecolor{violet}{RGB}{105, 5, 0}
\definecolor{brown}{RGB}{150, 50, 10}
\definecolor{green}{RGB}{5,100, 15}
\def\mr{\color{brown}}
\def\gr{\color{green}}
\def\vl{\color{violet}}


\def\mrA{{\mr\Aa}}
\def\mrB{{\mr\B}}
\def\mrC{{\mr\C}}
\def\mrD{{\mr\D}}
\def\grB{{\gr\B}}
\def\grC{{\gr\C}}
\def\grD{{\gr\D}}


% \thickmuskip=2mu plus 0.5mu minus 0.5mu
% \medmuskip=1mu plus 0.2mu minus 0.2mu
\def\vc{{\footnotesize VC}}
\def\nip{{\footnotesize NIP}}
\def\ip{{\footnotesize IP}}
\def\Fr{\mathop{\rm Fr}}
\def\Var{\mathop{\rm Var}}
\def\Ex{\mathord{\mathds E}}
% \def\Pr{\mathord{\mathds P}}


\def\ns{{}^*\kern-.3ex}
\def\cnonfork{\mathbin{\raise1.8ex\rlap{\kern0.6ex\rule{0.6ex}{0.1ex}}\rlap{\kern1.1ex\rule{0.1ex}{1.9ex}}\raise-0.3ex\hbox{$\smile$} } }

\def\simfin{\mathrel{\raise-.8ex\rlap{\kern0.9ex$\scriptscriptstyle\NN$}\sim\kern.5ex }}

\def\lessfin{\mathrel{\raise-.8ex\rlap{\kern0.3ex$\scriptscriptstyle\NN$}\raise.2ex\hbox{$<$} }}

\def\grtfin{\mathrel{\raise-.8ex\rlap{\kern0.6ex$\scriptscriptstyle\NN$}\raise.2ex\hbox{$>$} }}


\def\simpoly{\mathrel{\raise-.1ex\rlap{\kern0.6ex\tiny p}\raise.2ex\hbox{$\sim$} }}

\def\lesspoly{\mathrel{\raise-.3ex\rlap{\kern0.4ex\tiny p}\raise.2ex\hbox{$<$} }}

\def\grtpoly{\mathrel{\raise-.3ex\rlap{\kern0.7ex\tiny p}\raise.2ex\hbox{$>$} }}


\def\simlin{\mathrel{\raise-.3ex\rlap{\kern0.6ex\tiny l}\raise.2ex\hbox{$\sim$} }}

\def\lesslin{\mathrel{\raise-.5ex\rlap{\kern0.5ex\tiny l}\raise.2ex\hbox{$<$} }}

\def\grtlin{\mathrel{\raise-.3ex\rlap{\kern0.7ex\tiny l}\raise.2ex\hbox{$>$} }}

\def\eqPr{\mathrel{\raise-.6ex\rlap{\kern0.3ex\tiny Pr}\raise.1ex\hbox{$=$\,} }}


\begin{document}
\raggedbottom
% \renewcommand{\contentsname}{Pseudofinite Yoga}
% \tableofcontents


%%%%%%%%%%%%%%%%%%%%%%%%%%%%
%%%%%%%%%%%%%%%%%%%%%%%%%%%%
%%%%%%%%%%%%%%%%%%%%%%%%%%%%
%%%%%%%%%%%%%%%%%%%%%%%%%%%%
%%%%%%%%%%%%%%%%%%%%%%%%%%%%
%%%%%%%%%%%%%%%%%%%%%%%%%%%%
%%%%%%%%%%%%%%%%%%%%%%%%%%%%
\section{Pseudofinite yoga}
\def\medrel#1{\parbox[t]{5ex}{\hfil$\displaystyle #1$}}
\def\ceq#1#2#3{\parbox[t]{20ex}{$\displaystyle #1$}\medrel{#2}$\displaystyle  #3$}


As a matter of personal taste I have avoided the use of ultrapowers to deal with pseudofinite structures. 
%
Also, I have avoided the notion of pseudofinite dimension, as it is too general for the intended applications.


We say that a sentence holds in \emph{almost all finite models\/} if it holds in every finite model of sufficienly large cardinality.

Let $T_0$ be a given theory with arbitrarily large finite models.
%
We define 

\ceq{\hfill\emph{$T_{\rm fin}$}}{=}{\{\phi: M\models\phi\textrm{ for almost all finite } M\models T_0\}.}

A structure $N$ is a \emph{pseudofinite\/} model of $T_0$ if $N\models T_{\rm fin}$.
%
This is equivalent to requiring that every sentence $\phi\in\Th(N)$ holds in arbitrarily large finite models of $T_0$.
%
Note that every model of $T_{\rm fin}$ is infinite. (Other authors include in $T_{\rm fin}$ only sentences true in \textit{all\/} finite models. Hence they include the finite models among the pseudofinite ones.)

We now discribe a canonical expansion of a finite model $M$ to a two sorted structure $\<M,\NN\>$.
%
The language of $\<M,\NN\>$ is denoted by $L'$.
\begin{itemize}
  \item[1.] $L'$ expands the language of $M$ and $\NN$, where $\NN$ is considered as a structure in the language with symbols for all real relations and functions of any finite arity.
  \item[2.] $L'$ has a function $f_\phi:M^{|z|}\to\NN$ for each formula $\phi(x\,;z)\in L'$ and every finite tuples of variables $x\,;z$ of the first sort. 
  %
  The interpretation of $f_\phi(b)$ is the (finite) cardinality of $\phi(M\,;b)$.
\end{itemize}

For some application a richer language will be necessary (cfr.\@ Section~\ref{EH} below).
%
Other times we even need to expand $M$ with a few more sorts (cfr.\@ Section~\ref{samples} below).

We define

\ceq{\hfill\emph{$T'_{\rm fin}$}}{=}{\{\phi\in L': \<M,\NN\>\models\phi\textrm{ for all finite } M\models T_0\}}.

% Note that, for any given $M$, the interpretation of $A_{\phi}$ is not unique.
% Therefore the expansion $\<M,\NN\>$ is not unique.
% %
% We read $\<M,\NN\>\models\phi$ as claiming that every expansion as above models $\phi$.

\bigskip
In the sequel \emph{$\<\U,\ns\NN\>$\/} denotes some arbitrary saturated model of $T'_{\rm fin}$.

\begin{fact}
  Assume that $\psi\in L'$ holds in every $\<\U,\ns\NN\>$ as above.
  %
  Then $\<M,\NN\>\models\psi$ for all sufficienly large finite $M\models T_0$.\QED 
\end{fact}

Below we write \emph{$\ns|\phi(x\,;b)|$\/} for $f_\phi(b)$.
%
We call this the \emph{pseudofinite cardinality\/} of $\phi(x\,;b)$.
%
By the definition of  $T'_{\rm fin}$ it is clear that the pseudofinite cardinality of a definable set $\D$ does not depend on the formula defining it, so we can unambiguously write $\ns|\D|$.

We will need two preorder relations (a.k.a.\@ quasiorders) on ${}^*\kern-.2ex\NN\sm\{0,1\}$. We say that the fist \emph{linear\/} the second \emph{polynomial.} For $r,s\in {}^*\kern-.2ex\NN\sm\{0,1\}$
\begin{itemize}
  \item[1.]we write \emph{$r\leq_{\rm l }s$\/} if $r\le n\,s$ for some positive $n\in\NN$.
  \item[2.]we write \emph{$r\leq_{\rm p} s$\/} if $r\le s^n$ for some positive $n\in\NN$.
\end{itemize} 
The associated equivalence relation are denoted by\emph{$r\simlin s$\/} and \emph{$r\simpoly s$\/ }. 
%
Also, we write \emph{$r\lesslin s$\/} and \emph{$r\lesspoly s$\/} for the associated strict order, i.e.\@ if $n\,r<s$, respectively $r^n< s$ for all $n\in\NN$.

\begin{proposition}\label{prop_+=max}
  Let $\sim$ denote either $\simlin$ or $\simpoly$. 
  %
  For every non negative $r,s\in{}^*\kern-.2ex\NN$

  \ceq{\hfill r+s}
  {\sim}
  {\max\{r,\,s\}}

  In particular, if $\D, \C\subseteq\U^{z}$ are two definable sets, then

  \ceq{\hfill {}^*\kern-.2ex\big|\D\cup\C\big|}
  {\sim}
  {\max\big\{{}^*\kern-.2ex|\D|,\, {}^*\kern-.2ex|\C|\big\}.}
\end{proposition} 

\begin{proof}
  In fact 

\ceq{\hfill\max\{r,s\}}
{\le}
{r+s}

\ceq{}
{\le}
{2\max\{r,s\}}

\ceq{}
{\le}
{\big(\max\{r,s\}\big)^2}


This proves the first equivalence.
%
The second equivalence follows.
\end{proof}

Definable sets $\D\subseteq\U^x$ such that  $\ns|\D|\sim\ns|\U^x|$ are called \emph{large,} otherwise we say they are \emph{small.}
%
The context will clarify if we are referring to the linear or to the polynomial equivalence.


\begin{corollary}
  Let $\sim$ denote either $\simlin$ or $\simpoly$.
  %
  Let $p(x)\subseteq L(M)$ be a type such that $\phi(\U)$ is large for every $\phi(x)$ that is conjunction of formulas in $p(x)$.
  %
  Then $p(x)$ has an extension to a complete type with the same property.\QED
\end{corollary}

The following fact is immediate. It is stated to illustrate the typical application of the notions introduced above.

\begin{fact}\label{fact_application}
  Let $\sim$ denote either $\simlin$ or $\simpoly$.
  %
  Let $\phi(x\,;z)\in L$, where $x$ has finite length.
  % 
  Assume that in every model $\<\U,\ns\NN\>$ there is a $b\in\U^{z}$ such that $\phi(\U^x\,;b)$ is large.
  %
  Then there is an $n\in\NN$ such that every sufficiently large finite $M\models T_0$ contains a tuple $c$ such that

  \ceq{\hfill\big|M^x\big|}{\le}{n\,\big|\phi(M^x\,;c)\big|}\hfill if $\sim$ is $\simlin$\phantom{.}\kern25ex
  
  \ceq{\hfill\big|M^x\big|}{\le}{\big|\psi(M^x\,;c)\big|^n}\hfill if $\sim$ is $\simpoly$.\kern25ex\rlap{$\square$}

\end{fact}

The following two proposition only hold for $\simpoly$.

\begin{proposition}\label{prop_+=x}
  For every non negative $r,s\in{}^*\kern-.2ex\NN$

  \ceq{\hfill r\cdot s}
  {\simpoly}
  {\max\{r,s\}}
\end{proposition} 

\begin{proof}
  In fact 

  \ceq{\hfill\max\{r,s\}}
  {\le}
  {r\cdot s}
  
  \ceq{}
  {\le}
  {\big(\max\{r,s\}\big)^2}
\end{proof}

Note that, in particular, $\ns|\U|\simpoly\ns|\U^x|$ for all tuples of variables of finite arity.


\def\ceq#1#2#3{\parbox[t]{25ex}{$\displaystyle #1$}\medrel{#2}$\displaystyle  #3$}


\begin{corollary}\label{coroll_union_small}
  Let $\psi(x,z)\in L'$, where $x,z$ are finite tuples of variables of the first sort.
  %
  Let $\Aa\subseteq\U$.
  %
  Assume that 
  
  \ceq{\hfill\ns|\Aa|}{\lesspoly}{\ns|\U|}
  
  \ceq{\hfill\ns|\psi(a,\U^z)|}{\lesspoly}{\ns|\U|}\quad for every $a\in\Aa$.
  
  Then 
  
  \ceq{\hfill\strut^*\!\Big|\bigcup_{a\in\Aa}\psi(a,\U^z)\Big|}{\lesspoly}{\ns|\U|}.
\end{corollary}

\begin{proof}
  Let $r=\ns|\Aa|$.
  %
  Let $s=\ns|\psi(a,z)|$, where $a\in\Aa$ is choosen such that $s$ is maximal.
  %
  Then 
  
  \ceq{\hfill\strut^*\!\Big|\bigcup_{a\in\Aa}\psi(a,\U^z)\Big|}{\le}{r\cdot s}


  \ceq{}{\lesspoly}{\ns|\U|\cdot\ns|\U|}

  \ceq{}{\simpoly}{\ns|\U|.}
\end{proof}


%%%%%%%%%%%%%%%%%%
%%%%%%%%%%%%%%%%%%
%%%%%%%%%%%%%%%%%%
%%%%%%%%%%%%%%%%%%
\section{Erd\H{o}s-Hajnal for stable graphs}\label{EH}
\def\ceq#1#2#3{\parbox[t]{15ex}{$\displaystyle #1$}\medrel{#2}$\displaystyle  #3$}
In this section we assume that the language $L'$ contains a few more symbols (below, for convenience, all finite models considered have as domain a subset of $\omega$).
%
\begin{itemize}
  \item[3.]  Let $X$ be an $n$-ary relation symbol of the first sort. 
  %
  Let $\phi\in L\cup\{X\}$ be a sentence.
  %
  Then we require that $L'$ has an $n$-ary predicate $A_\phi$ in the first sort.
  %
  The interpretation of $A_\phi$ is a relation of maximal cardinality that makes $\phi$ true in $M$.
  %
  If more then one relation satisfy the requirement above, we choose the minimal one in the lexicographic order (recall that the domain of $M$ is a subset of $\omega$).
\end{itemize}
Let $r(\mbox{-},\mbox{-})$ be a binary relation symbol.
%
The theory that $T_0$ says that $r(\mbox{-},\mbox{-})$ is an irreflexive and symmetric relation (i.e.\@ a graph) and that it is $m$-stable, where $m$ is fixed but arbitrary.
For the purpose of these notes $m$-stable means that $R_\Delta(\U)\le m$, where \emph{$R_\Delta$\/} is defined below.

Let \emph{$\Delta$} $=\{r(a,x) :\, a\in\U\}$.
%
Let $\S\subseteq\U$.
%
We denote by $R_\Delta\big(\S\big)$ the Shelah binary rank of $\S$.
%
That is, the maximal height $n$ of binary tree of $\pmDelta$-formulas $\<\phi_i(x)\,:\, s\in {}^{<n}2\>$ such that for every $s\in {}^{n}2$ the type 

\ceq{\hfill }
{}
{\big\{\neg^{s_i}\,\phi_{s\restriction i}(x)\, :\, i\le n\big\}}

has a solution in $\S$.

As $T_0$ is stable, $R_\Delta\big(\U\big)$ is finite.
%
Among the large $L(\U)$-definable subsets of $\U$ we fix one, say \emph{$\S=\sigma(\U)$}, such that $R_\Delta\big(\S\big)$ is minimal.

\begin{fact}
  For every $\pmDelta$-definable set $\D\subseteq\U$, exactly one between $\S\cap\D$ and $\S\cap\neg\D$ is large.
\end{fact}

\begin{proof}
  As $\S$ is large, at least one between $\S\cap\D$ and $\S\cap\neg\D$ is large by Proposition~\ref{prop_+=max}.
  Now, suppose for a contradiction that they both are large.
  Let $n=R_\Delta\big(\S\big)$.
  By the minimality of $n$, both $\S\cap\D$ and $\S\cap\neg\D$ have rank $n$.
  This contradicts the definition of rank.
\end{proof}

\begin{theorem}
  There is a large set $\Aa\subseteq\U$ that is either a clique or an anticlique.
\end{theorem}

\begin{proof}
  Let $\S=\sigma(\U)$ be as defined above.
  %
  Let $p(x)\subseteq\pmDelta$ be maximally consistent among the $\pmDelta$-types such that $\sigma(x)\wedge\phi(x)$ is large for every conjunction of formulas in the type.
  %
  By the fact above, $p(x)$ is $\pmDelta$-complete, i.e.\@ either $\phi(x)$ or $\neg\phi(x)$ is in $p(x)$ for every $\phi(x)\in\pmDelta$.

  % Moreover, by compactness, $R_\Delta\big(\S\big)=R_\Delta\big(p(\U)\cap\S\big)$.

  Let $\B=\{a\in\S\ :\ r(a,x)\in p\}$.
  %
  By stability, $\B$ is definable in $L(\U)$.
  %
  At least one between $\B$ and $\S\sm\B$ is large.
  %
  Assume the first, the argument when $\S\sm\B$ is large is similar.

  Let $\Aa$ be a definable subsets of $\B$ that is maximal among those that satisfy the formula $\big(\A x,y\in\Aa\big)\,r(x,y)$.
  %
  The definition of $L'$ guarantees that such a set exists.
  
  We claim that $\Aa$ is large, hence it is the clique required by the theorem.
  %
  So, assume not, and reason for a contradiction.

  By the maximality of $\Aa$, for every $b\in\B\sm\Aa$ there is an $a\in\Aa$ such that $\neg r(a,b)$.
  %
  We rephrase this as the inclusion

  \ceq{\hfill\B\sm\Aa}
  {\subseteq}
  {\bigcup_{a\in\Aa}\neg r(a,\S)}

  By assumption $\Aa$ is not large, hence $\B\sm\Aa$ is large.
  %
  When $a\in\Aa$, in particular $r(a,x)\in p$.
  Therefore, by the fact above, $r(a,\S)$ is large and $\neg r(a,\S)$ is small.
  This is a contradiction by Corollary~\ref{coroll_union_small}.
\end{proof}

Finally the Erd\H{o}s-Hajnal property is obtained applying Fact~\ref{fact_application}.

\begin{corollary}
  There is an $n$ such that in every finite model $M\models T_0$ there is a set $A\subseteq M$ of cardinality at least $|M|^{1/n}$ that is either a clique or an anticlique.\QED
\end{corollary}


%%%%%%%%%%%%%%%%%%%%%%%%%%%%
%%%%%%%%%%%%%%%%%%%%%%%%%%%%
%%%%%%%%%%%%%%%%%%%%%%%%%%%%
%%%%%%%%%%%%%%%%%%%%%%%%%%%%
%%%%%%%%%%%%%%%%%%%%%%%%%%%%
\section{Pseudofinite counting measure}

\def\ceq#1#2#3{\parbox[t]{20ex}{$\displaystyle #1$}\medrel{#2}$\displaystyle  #3$}

Below, the theory $T_0$, the set of formulas $\Delta$, and the rank $m=R_\Delta$ are as defined the provious section.

Let $\D\subseteq\U$ be a definable set.
%
We define

\ceq{\hfill\Pr(\D)}
{=}
{\inf\Big\{\frac{m}{n}\ :\ m,n\in\NN\sm\{0\} \text{ such that } n\,\ns|\D|\le m\,\ns|\U|\Big\}},

where the infimum is taken in $\RR$.
%
It is immediate that $\Pr(\mbox{-})$ is a finite probability measure on the definable subsets of $\U$.
%
(By the Caratheodory Theorem it can be extended to a probability measure but this is not required here.)

We will write \emph{$\Aa\eqPr\D$} if $\Pr(\Aa\simdiff\D)=0$.

\begin{lemma}\label{lem_atoms}
  There are some $\{\wedge\}\pmDelta$-definable sets $\B_1,\dots,\B_n\subseteq\U$, where $n\le2^m$, such that for any $\{\wedge\}\pmDelta$-definable set $\Aa$, 
  
  \ceq{\hfill\Aa}{\eqPr}{\bigcup_{i\in I}\B_i,}

  per qualche $I\subseteq n$.
  %
  Also, we can require that all $\B_i$ have positive measure and that they form a partition $\U$.
\end{lemma}

\begin{proof}
  We prove a slightly more general claim.
  %
  Let $\D$ be a definable set of positive measure.
  %
  By induction on $R_\Delta(\D)=m$ we prove that there are $n\le 2^m$ many $\{\wedge\}\pmDelta$-definable sets $\B_i\subseteq\D$ such that for every $\{\wedge\}\pmDelta$-definable $\Aa\subseteq\D$ there is an $I\subseteq n$

  \ceq{\hfill\Aa}{\eqPr}{\bigcup_{i\in I}\B_i.}

  If  $R_\Delta(\D)=0$ then $\D$ is either disjoint of contained in every $\pmDelta$-definable set.
  So the claim holds trivially.

  Now, let $R_\Delta(\D)=m+1$.
  By the definition of binary rank, there is a $\Delta$-definable set $\B$ such that $R_\Delta(\D\cap\B)=R_\Delta(\D\cap\neg\B)=m$.
  Apply the induction hypothesis to $\D\cap\B$ if it has positive measure. 
  Do the same for $\D\cap\neg\B$.
  Note that at least one has positive measure.

  It is evident that the sets $\B_i$ obtained in the construction above are disjoint and have positive measure.
  Also, they cover $\U$ up to a set of measure $0$.
  Hence, replacing $\B_1$ with some $\B'_1\eqPr\B_1$ we obtain an actual cover of $\U$.
\end{proof}

\noindent\llap{\Large\color{red}\fontencoding{U}\fontfamily{futs}\selectfont\char 66\relax\ }%
Non sono sicuro questa qui sotto sia la corretta traduzione finita.

\begin{corollary}
  For every $\epsilon>0$ there is an $k$ such that for every finite $M\models T_0$ of cardinality larger than $k$, there is a partition of $M$, say $B_0,\dots,B_{n-1}\subseteq M$, where $n\le2^m$ such that $|B_i|>\epsilon\,|M|$ and for every $\{\wedge\}\pmDelta$-definable set $A\subseteq M$ there is an $I\subseteq n$ such that 

  \ceq{\hfill\bigg|A\simdiff\bigcup_{i\in I}B_i\bigg|}
  {<}
  {\epsilon\,|M|.}
\end{corollary}


%%%%%%%%%%%%%%%%%%%%%%%%%%%%%%%%%%%%%%%%
%%%%%%%%%%%%%%%%%%%%%%%%%%%%%%%%%%%%%%%%
%%%%%%%%%%%%%%%%%%%%%%%%%%%%%%%%%%%%%%%%
\section{Stable Szemer\'edi's regularity}

\def\ceq#1#2#3{\parbox[t]{15ex}{$\displaystyle #1$}\medrel{#2}$\displaystyle  #3$}
Below, the theory $T_0$, the set of formulas $\Delta$, and the rank $m=R_\Delta$ are as defined in Section~\ref{EH}.

\begin{lemma}
  There are two partitions of $\U$, say $\B_1,\dots,\B_n$, where $n\le 2^m$, and $\C_1,\dots,\C_{2^n}$, such that for all $i,j$ either $r(\B_i,\C_j)\eqPr\0$ or $r(\B_i,\C_j)\eqPr\B_i\times\C_j$.
\end{lemma}

\begin{proof}
  Let $\B_1,\dots,\B_n$ be the sets given by Lemma~\ref{lem_atoms}. For every $J\subseteq n$ let 
  
  \ceq{\hfill \C_J}{=}{\Big\{c\in\U\ :\ \ \bigcup_{i\in J}\B_i\;\eqPr r(\U,c)\Big\}}

  By Lemma~\ref{lem_atoms} if $i\in J$ then $r(\B_i,\C_J)\eqPr\B_i\times\C_J$, otherwise $r(\B_i,\C_J)\eqPr\0$.
\end{proof}

For $\Aa,\B\subseteq\U$, it is usual call \emph{density\/} the conditional probability

\ceq{\hfill\emph{$d(\Aa,\B)$}}
{=}
{\Pr\Big(r(\Aa,\B)\ \big|\ \Aa{\times}\B\Big)}

\ceq{}
{=}
{\frac{\Pr\big(r(\Aa,\B)\big)}{\Pr(\Aa{\times}\B)}}

This definition is given with the proviso that $\Aa$ and $\B$ have non zero probability.
%
The same notation is used for nonempty subsets $A,B\subseteq M$ of a finite model

\ceq{\hfill\emph{$d(A,B)$}}
{=}
{\frac{\big|r(A,B)\big|}{|A|\cdot|B|}}

\begin{corollary}
  There is a partition of $\U$ into sets of positive measure $\D_1,\dots,\D_n$, where $n\le2^{2^m+1}$ such that for all $i,j$ either $d(\D_i, \D_j)=0$ or $d(\D_i, \D_j)=1$.\QED
\end{corollary}


Finally, the the stable Szemer\'edi regularity lemma is obtained reasoning as in Fact~\ref{fact_application}.


\noindent\llap{\Large\color{red}\fontencoding{U}\fontfamily{futs}\selectfont\char 66\relax\ }%
%
Il corollario qui sotto sembra troppo forte.
Da quanto vedo in letteratura $n$ dovrebbe dipendere da $\epsilon$. 
Forse non \`e la traduzione al finito giusta.

\begin{corollary} 
  For every $\epsilon>0$ there is an $k$ such that for every finite $M\models T_0$ of cardinality larger than $k$, there is a partitions of $M$, say $D_1,\dots,D_n$, where $n\le2^{2^m+1}$ such that for all $i,j$ either $d(D_i,D_j)<\epsilon$ or $d(D_i,D_j)>1-\epsilon$.
  \QED
\end{corollary}

%%%%%%%%%%%%%%%%%%%%%%%%%%%%%%%%%%%%%
%%%%%%%%%%%%%%%%%%%%%%%%%%%%%%%%%%%%%
%%%%%%%%%%%%%%%%%%%%%%%%%%%%%%%%%%%%%
%%%%%%%%%%%%%%%%%%%%%%%%%%%%%%%%%%%%%
%%%%%%%%%%%%%%%%%%%%%%%%%%%%%%%%%%%%%
\section{My version of Radon-Nikodym}\label{samples}

\def\ceq#1#2#3{\parbox[t]{20ex}{$\displaystyle #1$}\medrel{#2}$\displaystyle  #3$}

\noindent\llap{\Large\color{red}\fontencoding{U}\fontfamily{futs}\selectfont\char 66\relax\ }%
Stupidaggini

In this section the canonical expansion of a finite model $M$ is a $3$-sorted structure $\<M,\NN, \NN^M\>$.
%
The language of $\<M,\NN, \NN^M\>$ is denoted (again) by $L'$.
%
The elements of $\NN^M$ are interpreted as a multisets and are called \emph{samples}.
%
If $s\in\NN^M$ and $s(a)=n$, the element $a$ occurs $n$-times in the multiset. 
\begin{itemize}
  \item[2'.] $L'$ has a function $f_\phi:\NN^{(M^{|x|})}\times M^{|z|}\to\NN$ for each formula $\phi(x\,;z)\in L'$ and every finite tuples of variables $x\,;z$ of the home sort. 
  %
  The interpretation of $f_\phi(r,b)$ is
  
  \ceq{\hfill f_\phi(r,b)}
  {=}
  {\sum_{a\in\phi(M,b)}r(a)}

  We will use the notation $\ns|\phi(r,b)|$ which is suggestive of is interpretation of cardinality in the multiset setting.
\end{itemize}

Let $\D\subseteq\U^x$ be a definable set.
%
Let $s\in\ns\NN^{(\U^x)}$1

%
We define

\ceq{\hfill{\Pr}_s(\D)}
{=}
{\inf\Big\{\frac{m}{n}\ :\ m,n\in\NN\sm\{0\} \text{ such that } n\,\ns|s\cdot\Indicator_\D|\le m\,\ns|s|\Big\}},

where the infimum is taken in $\RR$.
%
It is immediate that ${\Pr}_s(\mbox{-})$ is a finite probability measure on the definable subsets of $\U$.
%
(By the Caratheodory Theorem it can be extended to a probability measure but this is not required here.)


We say that $r,t\in\ns\NN^{(\U^{x})}$ are \emph{$\Delta$-equivalent samples\/} if for every $\psi(x)\in\pmDelta$\smallskip

\ceq{\hfill\frac1n}
{>}
{\bigg|\frac{\ns|\psi(r)|}{\ns|r|}-\frac{\ns|\psi(t)|}{\ns|t|}\bigg|}\hfill for all $n\in\NN\sm\{0\}$.\smallskip

Let $x^{\rm s}$ be a variable of the sort of $\ns\NN^{(\U^x)}$.
Let 

\ceq{\hfill\emph{$\Delta^{\rm\!s}$}}{=}{\Big\{m{\cdot}\ns|\psi(x^{\rm s})|\le n{\cdot}\ns|x^{\rm s}|\ :\  m,n\in\NN\ \mathrm{ and }\ \psi(x)\in\{{\wedge}\}\pmDelta\Big\}.}

The samples that realize the type $p(x^{\rm s})=\Delta^{\rm\!s}\kern-.2ex\mbox{-}\tp(r)$ are those that are $\Delta$-equivalent to $r$. 



\end{document}
